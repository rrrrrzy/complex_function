\documentclass[12pt, a4paper, oneside]{ctexart}
%\usepackage[UTF8]{ctex}
\usepackage{circuitikz}         %一个很好的电路绘制包体
\ctikzset{logic ports=ieee}     %所有逻辑门使用IEEE标准
\usetikzlibrary{calc}  
\usepackage{bm}

\usepackage{amsmath}
\usepackage{amsthm}
\theoremstyle{plain}
\newtheorem{theorem}{定理}[section]
\newtheorem{definition}{定义}[section]
\newtheorem{nature}{性质}
\theoremstyle{definition}
\newtheorem{example}{例}
\theoremstyle{definition}
\newtheorem*{solution}{解}

\title{复变函数与积分变换}
\author{rrrrrzy}
\begin{document}
\maketitle
\section{复数的基本运算}
\subsection{复数的表示}
先定义一个复数为$z = x + iy$, 记它的实部为$\Re(z)=x$, 虚部为$\Im(z)=y$, 在复平面上表示为(x, y), 当$z\neq 0$时, 
称从正实轴到$z$的终边的角的弧度数$\theta$为$z$的辐角, 记作 $\text{\text{Arg}z}=\theta $
, 此时有$\tan (\text{Arg}z) = \frac{y}{x}$, 因为一个不为零复数有无数个辐角, 
如果$\theta_{1}$时其中的一个, 则$\text{Arg}z=\theta_{1}+2k\pi(k\in N)$, 其中
满足$-\pi<\theta_{0}\leq \pi$的角称为$\text{Arg}z$的主值, 记作$\theta_{0}=\arg z$.

我们规定, 当$z=0$时, 它的辐角不确定.

此外, 一个复数还可以表示为$z=r(\cos\theta+i\sin\theta)$, 这就是复数的三角表示式.

又由欧拉公式: $e^{i\theta}=\cos\theta+i\sin\theta$, 有: $z=re^{i\theta}$, 
这就是复数的指数表示式.

\subsection{复数的幂与根}
n个相同复数$z$的乘积记作$z^{n}$, 有: $z^{n}=r^{n}(\cos n\theta+i\sin n\theta)$, 浅作证明如下:

\begin{proof}
已知一个复数的指数表示式为$z=re^{i\theta}$, 则它的n次方为: $z^{n}=r^{n}e^{i\cdot n\theta}$, 认为$n\theta$
是一个整体, 展开为三角表示式: $z^{n}=r^{n}(\cos n\theta+i\sin n\theta)$

\end{proof}

特别的, 当复数的模$r=1$时, 有De Moivre公式: 
\[
    (\cos\theta+i\sin\theta)^{n}=\cos n\theta+i\sin n\theta
\]

下面讨论复数的根:

记复数的n次根为$w=\sqrt[n]{z}=r^{\frac{1}{n}}(\cos\frac{\theta+2k\pi}{n}+i\sin\frac{\theta+2k\pi}{n})$, 简作证明如下
\begin{proof}
为了求出根$w$, 我们令
\[
    z=r(\cos\theta+i\sin\theta),\; w=\rho (\cos\varphi+i\sin\varphi)
\]
由 De Moivre公式有 $\rho^{n}(\cos n\varphi+i\sin n\varphi)=r(\cos\theta+i\sin\theta) $

于是有 $\rho^{n}=r,\; \cos n\varphi=\cos\theta,\; \sin n\varphi=\sin\varphi$, 显然, 后两式成立的充要条件是 
$n\varphi=\theta+2k\pi,\; (k=0, \pm1, \pm2,...)$. 由此有$\rho=r^{\frac{1}{n}},
 \;\varphi=\frac{\theta+2k\pi}{n}$, 证毕
    
\end{proof}

\subsection{其他重要的定义}
\begin{definition}[曲线光滑的定义]

    设曲线$z(t)=x(t)+iy(t)$可以表示为$z=z(t)$, 那么, 如果在区间$a\leq t \leq b$上, $x'(t), y'(t)$都是连续的, 且
对$t$的每一个值都有$[x'(t)]^{2}+[y'(t)]^{2}\neq0$, 则称该曲线是光滑的.
\end{definition}


\begin{definition}[曲线简单和开闭的定义]
    设一段曲线自身没有交叉点, 则称它为简单曲线. 如果起点与终点重合, 称它为简单闭曲线
\end{definition}

\begin{definition}[单联通域和多联通域的定义]
    复平面上有一区域 B, 如果在其中任作一条简单闭合曲线, 曲线的内部总属于 B, 就称为单联通域. 一个区域
如果不是单联通域, 就称为多联通域.
\end{definition}

\section{复变函数的基本操作}
\subsection{复变函数的极限}

每个复变函数都可以表示为$f(z)=u(x)+v(x)\cdot i$, 则对它求极限有以下几种定理

\begin{theorem}[定理一]
    设$f(z)=u(x,y)+iv(x,y),\; A=u_{0}+iv_{0},\; z_{0}=x_{0}+iy_{0}$, 则$\lim_{z\to z_{0}}f(z)=A$的充要条件
    是
    \[
        \lim_{ (x,y)\to(x_{0},y_{0}) } u(x,y)=u_{0} \quad \lim_{(x,y)\to(x_{0},y_{0}) }v(x,y)=v_{0}
    \]
\end{theorem}

这个定理把求一个复变函数的极限问题转化为了求两个二元实变函数的极限问题.

与此相同, 实变函数中的极限基本运算规律在复变函数中也是适用的
\begin{theorem}[定理二]
    如果$\lim_{z\to z_{0}}f(z)=A,\;\lim_{z\to z_{0}}g(z)=B$, 则有
    \begin{align}
        \lim_{z\to z_{0}} [f(z)\pm g(z)] &= A \pm B; \\
        \lim_{z\to z_{0}} f(z)g(z) &= AB;\\
        \lim_{z\to z_{0}} \frac{f(z)}{g(z)} &= \frac{A}{B} \quad (B\neq0)
    \end{align}   

\end{theorem}
\subsection{复变函数的导数}

\subsubsection{复变函数导数的定义}
\begin{definition}[复变函数可导的定义]
    设函数$\omega=f(z)$定义于区域$D$.$z_{0}$是$S$中一点, 点$z_{0}+\Delta z$不出$D$的范围. 如果极限
    $\lim_{\Delta z\to0}\frac{f(z_{0}+\Delta z)-f(z_{0})}{\Delta z}$存在, 那么就称$f(z)$在$Dz_{0}$
    可导.这个极限值就称作$f(z)$在$z_{0}$的导数, 记作$f'(z_{0})$
\end{definition}

可见, 复变函数导数的定义和一元实变函数的定义类似, 只是$\Delta z\to 0$的方式是任意的, 不再局限于数轴上的双向趋近. 
因此, 导数也就没有了左导数和右导数之说, 这一点又和二元实变函数的导数类似.

\subsubsection{复变函数的连续性}
与实变函数相同, 复变函数\textbf{可导一定连续, 连续不一定可导}

\subsubsection{复变函数的求导法则}
与一元实变函数的求导公式在形式上完全一致

\subsubsection{复变函数的微分}
与一元实变函数的微分概念完全一致, 需要特别指出的是:函数$\omega=f(z)$在$z_{0}$处可导与在$z_{0}$处可微是等价的

\subsection{解析函数}
\subsubsection{解析函数的定义}
\begin{definition}
    如果函数$f(z)$在$z_{0}$及$z_{0}$的邻域内处处可导, 那么称$f(z)$在$z_{0}$解析.如果$f(z)$在区域$D$内每一点
    解析, 那么称$f(z)$在$D$内解析, 或称$f(z)$是$D$内的一个解析函数(全纯函数或正则函数).\\
    如果$f(z)$在$z_{0}$上不解析, 那么称$z_{0}$为$f(z)$的奇点.
\end{definition}
由此可见, 函数在区域内解析和在区域内可导是等价的, 但函数在一点处可导, 不一定在该点处解析
\subsubsection{函数解析的充要条件}
\begin{theorem}[函数可导的充要条件]
设函数$f(z)=u(x,y)+iv(x,y)$定义在区域$D$内, 则$f(z)$在$D$内一点$z=x+iy$可导的充要条件是$u(x,y)$与$v(x,y)$在
点$(x,y)$可微, 并且在该点满足柯西-黎曼方程
\[
    \frac{\partial u}{\partial x}=\frac{\partial v}{\partial y}, \quad\frac{\partial u}{\partial y}=
    -\frac{\partial v}{\partial x}.
\]
\end{theorem}
特别强调, 复变函数$z=u+iv$在一点可导与可微是等价的, 而柯西-黎曼条件是说, 构成这个复变函数的实部和虚部的实函数要可微
,并不是这个复变函数本身可微.

立即的, 从柯西-黎曼方程可以推出函数$f(z)=u(x,y)+iv(x,y)$在点$z=x+iy$的导数公式:
\[
    f'(z)=\frac{\partial u}{\partial x}+i\frac{\partial v}{\partial x}=
    \frac{1}{i}\frac{\partial u}{\partial y}+\frac{\partial v}{\partial y}
\]
根据函数解析的定义及定理2.3, 得到以下定理
\begin{theorem}[函数在区域$D$解析的充要条件]
    函数$f(z)=u(x,y)+iv(x,y)$在定义域$D$内解析的充要条件是:$u(x,y)$与$v(x,y)$在$D$内可微, 并且满足柯西-黎曼方程
\end{theorem}
\subsection{几个特殊的复变函数}
\subsubsection{指数函数}
\begin{definition}[复变函数指数的定义]
    函数$f(z)=e^x(\cos y+i\sin y)$是一个在复平面内处处解析的函数, 且有$f'(z)=f(z)$, 当$\Im(z)=y=0$时, $f(z)=e^x$, 
    这个函数满足复变函数指数的定义, 记作$\exp z=e^x(\cos y+i\sin y)$, 称其为复变函数$z$的指数函数.
\end{definition}
\noindent 该定义等价于关系式:
\begin{align*}
    &\left\lvert \exp z\right\rvert =e^z\\
    &\text{Arg}(\exp z)=y+2k\pi, \quad\text{其中, $k$为任何整数, }
\end{align*}
\noindent 显然$\exp z$也服从加法定理:$\exp z_1\cdot\exp z_2=\exp (z_1+z_2)$

方便起见, 通常用$e^z$代替$\exp z$, 但这里的$e^z$没有幂的意义, 仅作为代替符号使用, 故有
\[
e^{z}=e^x(\cos y+i\sin y).
\]
那么由加法定理可以推出$\exp z$的周期为$2k\pi i$, 即
\[
e^{z+2k\pi i} = e^z\cdot e^{2k\pi i}=e^z, \quad \text{其中$k$为任何整数}
\]
这个性质只有复变函数中才存在

\subsubsection{对数函数}
\begin{definition}[对数函数的定义]
和实变函数一样, 对数函数定义为指数函数的反函数. 我们把满足方程$e^\omega = z,\;(z\neq0)$的函数
$\omega=f(z)$称为对数函数
\end{definition}
令$\omega=u+iv,z=re^{i\theta}$, 那么$e^{u+iv}=re^{i\theta}$, 因此有
\[
\omega=\ln \left\lvert z\right\rvert +i\text{Arg}z.
\]

由于Arg$z$是多值函数, 所以对数函数$\omega=f(z)$是多值函数, 并且每两个值相差$2\pi i$的整数倍, 记作
\[
\text{Ln}z=\ln\left\lvert z\right\rvert +i\text{Arg}z.
\]
如果规定上式中的Arg$z$取主值$\arg z$, 则Ln$z$为一单值函数, 记为$\ln z$, 称作Ln$z$的主值. 因此就有
\[
\ln z= \ln \left\lvert z\right\rvert + i\arg z
\]
其余各值可以由Ln$z=\ln z+2k\pi i \quad(k=\pm1,\pm2,...)$表达

利用辐角的相应性质不难证明, 复变对数函数保持了实变对数函数的基本性质:
\begin{align*}
    \text{Ln}(z_1z_2) &= \text{Ln}z_1+\text{Ln}z_2,\\
    \text{Ln}\frac{z_1}{z_2} &= \text{Ln}z_1-\text{Ln}z_2
\end{align*}
   但需要注意的是, 下面的式子不再成立
\begin{align*}
    \text{Ln}z^n &= n\text{Ln}z,\\
    \text{Ln}\sqrt[n]{z} &= \frac{1}{n}\text{Ln}z
\end{align*}
也很容易证明, $\ln z$在除去原点及负实轴的平面内解析, Ln$z$的各个分支在除去
原点及负实轴的平面内也解析, 并且有相同的导数值

\subsubsection{乘幂$a^b$与幂函数}

\section{复变函数的积分}
\subsection{积分存在的条件及其计算方法}
\subsubsection{积分存在的条件}
当$f(z)$是连续函数而$C$是光滑曲线时, 积分$\int_{C}f(z)  \,dz $是一定存在的
\subsubsection{积分的几个基本计算法}
对于普通的线积分:
\begin{align*}
    \int_{C}f(z)\,dz=\int_{C}(u+iv)\,(dx+idy) &= \int_{C}u\,dx+iv\,dx+iu\,dy-v\,dy \\
    &= \int_{C}u\,dx-v\,dy+i\int_{C}v\,dx+\,udy
\end{align*}

对于曲线$C$可以化为参数方程$z=z(t),\,\alpha\leq t\leq\beta$的积分:
\[
    \int_{C}f(z)\,dz=\int_{\alpha}^{\beta}f[z(t)]z'(t)\,dt
\]
\subsubsection{一个常用的积分}
\begin{example}
    计算$\oint_{C}\frac{dz}{(z-z_{0})^{n+1}}$, 其中$C$为以$z_{0}$为中心, $r$为半径的正向圆周, $n$为整数
\end{example}
\begin{solution}
    $C$的方程可以写作$z=z_{0}+re^{i\theta},0\leq\theta\leq2\pi$, 所以
    \[
        \oint_{C}\frac{dz}{(z-z_{0})^{n+1}}=\int_{0}^{2\pi}\frac{ire^{i\theta}}{r^{n+1}e^{(n+1)i\theta}}\,d\theta
        =\int_{0}^{2\pi}\frac{i}{r^{n}e^{in\theta}}\,d\theta=\frac{i}{r^{n}}\int_{0}^{2\pi}e^{-in\theta}\,d\theta
    \]

    则当$n=0$时, 结果为: $i\int_{0}^{2\pi}\,d\theta=2\pi i$,

    当$n\neq0$时, 结果为: $\frac{i}{r^{n}}\int_{0}^{2\pi}(\cos n\theta-i\sin n\theta)\,d\theta=0$.
\end{solution}
    \noindent 因此得到: 
    \[
        \oint_{C}\frac{dz}{(z-z_{0})^{n+1}}=
        \begin{cases}
            2\pi i, & n=0\\
            0, &n\neq 0
        \end{cases}
    \]
    其中, $C$只要是包围$z_{0}$的正向简单曲线即可, 它的特点是积分与路线中心和周围半径无关
\subsection{柯西-古萨基本定理}
从对普通线积分的计算可以发现, 一个复变函数的积分被转化为了两个二元实变函数的第二类曲线积分, 要计算它们可以使用格林公式:
\begin{align*}
    \int_{C}u\,dx-v\,dy &= \iint_{D}(-\frac{\partial v}{\partial x}-\frac{\partial u}{\partial y})\,dxdy \\
    \int_{C}v\,dx+\,udy &= \iint_{D}(\frac{\partial u}{\partial x}-\frac{\partial v}{\partial y})\,dxdy
\end{align*}
由定理2.3得到的导数公式知:
\begin{align*}
\frac{\partial u}{\partial x} &= \frac{\partial v}{\partial y}\\
\frac{\partial v}{\partial x} &= -\frac{\partial u}{\partial y}
\end{align*}
这刚好满足积分与路径无关的充要条件:
\[
    \frac{\partial P}{\partial y}=\frac{\partial Q}{\partial x}
\]
因此, 得到柯西-古萨定理
\begin{theorem}[柯西-古萨定理]
    如果函数$f(z)$在单联通域$B$内处处解析, 那么函数$f(z)$沿$B$内的任何一条封闭曲线$C$的积分为零:
    \[
    \oint_{C}f(z)\,dz=0
    \]
\end{theorem}
定理中的$C$可以不是简单曲线, 而定理成立的条件之一就是曲线$C$要属于区域$B$. 另外, 如果$C$是区域$B$的边界,
$f(z)$在$B$内解析, 在闭区域$\bar{B}$上连续, 定理仍然成立

\subsection{基本定理的推广——复合闭路定理}
如果曲线$C$的内部不完全包含于$D$时, 柯西-古萨定理不一定成立, 这里给出复合闭路定理
\begin{theorem}[复合闭路定理]
    \hspace{0em}
    
    设$C$是多联通域$D$内的一条简单闭曲线, $C_1,C_2,...,C_n$是在$C$内部的简单闭曲线, 它们互不包含也互不相交, 
并且以$C_1,C_2,...,C_n$为边界的区域全含于$D$, 如果$f(z)$在$D$内解析, 那么: 

$\oint_{C}f(z)\,dz=\sum_{k=1}^{n}\oint_{C_k}f(z)\,dz$, 其中$C$和$C_k$均取正方向; 

$\oint_{\Gamma}f(z)\,dz=0$, 这里$\Gamma$为由$C$及$C_k(k=1,2,...,n)$所组成的复合闭路
(其方向是:$C$按逆时针,$C_k^n$按顺时针)

\end{theorem}
\subsection{原函数与不定积分}
我们知道, 线积分沿封闭曲线的积分为零跟曲线积分与路径无关是两个等价的概念, 因此从柯西-古萨定理有:
\begin{theorem}
    如果函数$f(z)$在单联通域$B$内处处解析, 那么积分$\int_Cf(z)\,dz$与连接起点和终点的路线$C$无关
\end{theorem}
从定理3.3有, 解析函数在单联通域内的积分只和起点$z_0$及终点$z_1$有关, 因此有
\[
\int_{C_1}f(z)\,dz=\int_{C_2}f(z)\,dz=\int_{z_0}^{z_1}f(z)\,dz
\]
固定$z_0$, 让$z_1$在$B$内变动, 并令$z=z_1$, 则变上限积分$\int_{z_0}^{z}f(\zeta)\,d\zeta$在
$B$内确定了一个单值函数$F(z)$, 即
\[
F(z)=\int_{z_0}^{z}f(\zeta)\,d\zeta.
\]
对这个函数, 我们有
\begin{theorem}
    如果$f(z)$在单联通域$B$内处处解析, 那么函数$F(z)$必为$B$内的一个解析函数, 并且$F'(z)=f(z)$
\end{theorem}
\noindent 这个定理与一元实变函数的变上限积分的求导定理是完全类似的, 因此也能类似地引入积分基本定理和
牛顿-莱布尼兹公式.
\begin{theorem}
    如果$f(z)$在单联通域$B$内处处解析, $G(z)$为$f(z)$的一个原函数, 那么
    \[
        \int_{z_0}^{z_1}f(z)\,dz=G(z_1)-G(z_0),
    \]
    这里的$z_0,z_1$为域$B$内的两点.
\end{theorem}
\subsection{柯西积分公式}
设有一函数$f(z)$在$B$内解析, $z_0$是$B$内一点, 显然$\frac{f(z)}{z-z_0}$在
$z_0$不解析, 那么积分$\oint_{C}\frac{f(z)}{z-z_0}\,dz$一般不为零, 由闭路变形原理知, 
该积分的值沿任何一条围绕$z_0$的闭合曲线都是相同的, 不妨从以$z_0$为圆心的圆去逼近它,即$z\to z_0$, 最终有
\[
\oint_{C}\frac{f(z)}{z-z_0}\,dz=\oint_{C}\frac{f(z_0)}{z-z_0}\,dz=f(z_0)\oint_{C}\frac{dz}{(z-z_0)^1}=2\pi if(z_0)
\]
这样就得到了下面的定理
\begin{theorem}[柯西积分公式]
    如果$f(z)$在区域$D$内处处解析, $C$为$D$内的任何一条正向简单闭曲线, 它的内部完全包含于$D$, $z_0$为$C$内的任一点, 那么有
    \[
        f(z_0)=\frac{1}{2\pi i}\oint_{C}\frac{f(z)}{z-z_0}\,dz
    \]
\end{theorem}
这个公式可以把函数在$C$内任一点的值用它在边界上的值来表示. 换句话说, 如果$f(z)$在区域边界上的值一经确定, 那么它在区域内部
任一点的值也就确定了, 这是解析函数的又一特征

如果$C$是圆周$z=z_0+Re^{i\theta}$, 那么上式就可以写作
\begin{align*}
f(z_0) &= \frac{1}{2\pi i}\int_{0}^{2\pi}\frac{f(z_0+Re^{i\theta})}{Re^{i\theta}}\,d(z_0+Re^{i\theta})\\
&= \frac{1}{2\pi}\int_{0}^{2\pi}f(z_0+Re^{i\theta})\,d\theta
\end{align*}
\noindent 也就是说, \textbf{一个解析函数在圆心处的值等于它在圆周上的平均值}

\subsection{解析函数的高阶导数}
一个解析函数不仅有一阶导数, 而且还有高阶导数, 它的值也可以用函数在边界上的值通过积分来表示, 我们有以下定理
\begin{theorem}
    解析函数$f(z)$的导数仍为解析函数, 它的$n$阶导数为:
    \[
        f^{(n)}(z_0)=\frac{n!}{2\pi i}\oint_{C}\frac{f(z)}{(z-z_0)^{(n+1)}}\,dz \quad (n=1,2,...)
    \]
    其中$C$为在函数$f(z)$的解析区域$D$内围绕$z_0$的任何一条正向简单闭曲线, 而且它的内部全包含于$D$
\end{theorem}
高阶导数公式的作用不在于通过积分来求导, 而是在于通过求导来求积分

\subsection{解析函数与调和函数的关系}
\begin{definition}
    如果二元实变函数$\varphi(x,y)$在区域$D$内有二阶连续偏导数, 并且满足
    拉普拉斯(Laplace)方程
    \[
    \frac{\partial^2\varphi}{\partial x^2}+\frac{\partial^2\varphi}{\partial y^2}=0
    \]
    则称$\varphi(x,y)$是区域$D$内的调和函数
\end{definition}
\begin{theorem}
    任何在区域$D$内解析的函数, 它的实部和虚部都是$D$内的调和函数
\end{theorem}
这个定理说明了调和函数和解析函数的关系
\begin{definition}
    设$u=(x,y)$是区域$D$内给定的调和函数, 我们把使$u+iv$在$D$内构成解析函数
    的调和函数$v(x,y)$称为$u(x,y)$的共轭调和函数
\end{definition}
这个定义等价于: 在$D$内满足柯西-黎曼方程$u_x=v_y,\,v_x=-u_y$的两个调和函数中, 
$v$称为$u$的共轭调和函数.

因此, 上面的定义说明了, 区域$D$内的解析函数的虚部是实部的共轭调和函数.

...(待补充: 用偏积分法求已知实部的复变函数)

\section{级数}
\subsection{复数项级数}
\begin{definition}[复数列的极限]
    设$\left\{ \alpha_n \right\}$为一复数列, 其中$\alpha_n=a_n+ib_n$, 又设
    $\alpha=a+bi$是一个确定的复数, 如果给定任意的$\epsilon>0$, 相应的能找到一个
    正数$N(\epsilon)$, 使得$\left\lvert \alpha_n -\alpha \right\rvert < \epsilon$
    在$n>N$时成立, 那么$\alpha$称作复数列$\left\{ \alpha_n \right\}$当
    $n\to\infty $时的极限, 记作
    \[
        \lim_{n\to\infty}\alpha_n=\alpha
    \]
\end{definition}
\noindent 仿照复变函数极限的求法, 我们立即得到求得复数列收敛的充要条件
\begin{theorem}
    复数列$\left\{\alpha_n\right\}(n=1,2,...)$收敛于$\alpha$的充要条件是
    \[
    \lim_{n\to\infty}a_n=a,\,\lim_{n\to\infty}b_n=b.
    \]
\end{theorem}
在复数域内, 级数的定义和实数域也很相似
\begin{definition}[复数域的级数]
    设$\left\{\alpha_n\right\}=\left\{a_n+b_n\right\}$为一复数列, 表达式
    \[
    \sum_{n=1}^{\infty}\alpha_n=\alpha_1+\aleph_2+\dots+\alpha_n+\dots
    \]
    称作无穷级数, 其最前面$n$项的和$s_n=\displaystyle\alpha_a+\alpha_2+\dots+\alpha_n$
    称为级数的部分和, 如果部分和数列$\left\{s_n\right\}$收敛, 那么其对应的级数
    收敛, 并且极限$\displaystyle\lim_{n\to\infty}s_n=s$称为级数的和, 如果部分和数列不收敛,
    那么级数发散
\end{definition}
\begin{theorem}
    级数$\displaystyle\sum_{n=1}^{\infty}\alpha_n$收敛的充要条件是级数
    $\displaystyle\sum_{n=1}^{\infty}a_n,\, \sum_{n=1}^{\infty}b_n$都收敛
\end{theorem}
定理4.2将复数项级数的审敛问题转化为实数项级数的审敛问题, 由实数项级数
$\displaystyle\sum_{n=1}^{\infty}a_n,\, \sum_{n=1}^{\infty}b_n$收敛的必要条件
\[
\displaystyle\lim_{n\to\infty}a_n=0,\,\displaystyle\lim_{n\to\infty}b_n=0
\]
立即可以得到$\displaystyle\lim_{n\to\infty}\alpha_n=0$, 从而得到复数项级数
$\displaystyle\sum_{n=1}^{\infty}\alpha_n$收敛的必要条件是$\displaystyle\lim_{n\to\infty}\alpha_n=0$
\begin{theorem}
    如果$\displaystyle\sum_{n=1}^{\infty}\left\lvert \alpha_n\right\rvert $收敛, 那么
    $\displaystyle\sum_{n=1}^{\infty}\alpha_n$也收敛, 且不等式
    $\displaystyle\left\lvert \sum_{n=1}^{\infty}\alpha_n\right\rvert \leq\displaystyle\sum_{n=1}^{\infty}\left\lvert \alpha_n\right\rvert$
    成立
\end{theorem}
如果$\displaystyle\sum_{n=1}^{\infty}\left\lvert \alpha_n\right\rvert $收敛, 那么称级数
$\displaystyle\sum_{n=1}^{\infty}\alpha_n$为绝对收敛, 非绝对收敛的收敛级数称作条件收敛级数, 
$\displaystyle\sum_{n=1}^{\infty}\alpha_n$绝对收敛的充要条件是级数
$\displaystyle\sum_{n=1}^{\infty}a_n,\, \displaystyle\sum_{n=1}^{\infty}b_n$绝对收敛
\subsection{幂级数}
\begin{definition}[幂级数的定义]
    
\end{definition}


\end{document}